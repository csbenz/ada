%%%%%%%%%%%%%%%%%%%%%%%%%%%%%%%%%%%%%%%%%%%%%%%%
% Python Cheat Sheet
% baposter Landscape Poster
% LaTeX Template
% Version 1.0 (11/06/13)
% baposter Class Created by:
% Brian Amberg (baposter@brian-amberg.de)
% This template has been downloaded from:
% http://www.LaTeXTemplates.com
% License:
% CC BY-NC-SA 3.0 (http://creativecommons.org/licenses/by-nc-sa/3.0/)
% Edited by Michelle Cristina de Sousa Baltazar
%%%%%%%%%%%%%%%%%%%%%%%%%%%%%%%%%%%%%%%%%%%%%%%%

%----------------------------------------------------------------
%   PACKAGES AND OTHER DOCUMENT CONFIGURATIONS
%----------------------------------------------------------------

\documentclass[landscape,a0paper,fontscale=0.285]{baposter} % Adjust the font scale/size here
\title{To be in season or not to be in season}
\usepackage[brazilian]{babel}
\usepackage[utf8]{inputenc}

\usepackage{graphicx} % Required for including images
\graphicspath{{figures/}} % Directory in which figures are stored

\usepackage{xcolor}
\usepackage{colortbl}
\usepackage{tabu}

\usepackage{mathtools}
%\usepackage{amsmath} % For typesetting math
\usepackage{amssymb} % Adds new symbols to be used in math mode

\usepackage{booktabs} % Top and bottom rules for tables
\usepackage{enumitem} % Used to reduce itemize/enumerate spacing
\usepackage{palatino} % Use the Palatino font
\usepackage[font=small,labelfont=bf]{caption} % Required for specifying captions to tables and figures

\usepackage{multicol} % Required for multiple columns
\setlength{\columnsep}{1.5em} % Slightly increase the space between columns
\setlength{\columnseprule}{0mm} % No horizontal rule between columns

\usepackage{tikz} % Required for flow chart
\usetikzlibrary{decorations.pathmorphing}
\usetikzlibrary{shapes,arrows} % Tikz libraries required for the flow chart in the template

\newcommand{\compresslist}{ % Define a command to reduce spacing within itemize/enumerate environments, this is used right after \begin{itemize} or \begin{enumerate}
\setlength{\itemsep}{1pt}
\setlength{\parskip}{0pt}
\setlength{\parsep}{0pt}
}

\definecolor{lightblue}{rgb}{0.145,0.6666,1} % Defines the color used for content box headers

\begin{document}

\begin{poster}
{
headerborder=closed, % Adds a border around the header of content boxes
colspacing=0.8em, % Column spacing
bgColorOne=white, % Background color for the gradient on the left side of the poster
bgColorTwo=white, % Background color for the gradient on the right side of the poster
borderColor=lightblue, % Border color
headerColorOne=black, % Background color for the header in the content boxes (left side)
headerColorTwo=lightblue, % Background color for the header in the content boxes (right side)
headerFontColor=white, % Text color for the header text in the content boxes
boxColorOne=white, % Background color of the content boxes
textborder=roundedleft, % Format of the border around content boxes, can be: none, bars, coils, triangles, rectangle, rounded, roundedsmall, roundedright or faded
eyecatcher=true, % Set to false for ignoring the left logo in the title and move the title left
headerheight=0.15\textheight, % Height of the header
headershape=roundedright, % Specify the rounded corner in the content box headers, can be: rectangle, small-rounded, roundedright, roundedleft or rounded
headerfont=\Large\bf\textsc, % Large, bold and sans serif font in the headers of content boxes
%textfont={\setlength{\parindent}{1.5em}}, % Uncomment for paragraph indentation
linewidth=2pt % Width of the border lines around content boxes
}
%----------------------------------------------------------------
%   TÍTULO
%----------------------------------------------------------------
{\bf\textsc{To be in season or not to be in season}\vspace{0.5em}} % Poster title
{\textsc{ To be in season or not to be in season \hspace{20pt}}}
{\textsc{ Kate Dopiro, Fabien Zellweger \& Christopher Benz \\ \includegraphics[scale=0.3]{img/EPFL-Logo-RVB-55.jpg} \hspace{6pt}}} 


%------------------------------------------------
% BÁSICO DO PYTHON
%------------------------------------------------
\headerbox{Data Scraping:}{name=objectives,column=0,row=0}{

%--------------------------------------
\colorbox[HTML]{CCFFFF}{\makebox[\textwidth-2\fboxsep][l]{\bf - Introduction:}}
Humankind has acquired knowledge about food for ages. As a result we know that each food has its season. However with the rise of civilization we are no longer dependent on it as any food can be transported to any point in the world. The question is then: how much food do we consume out of the season? Our project focuses on two aspects: the season the foods are naturally produced, and the area where they are naturally produced.


%--------------------------------------
\colorbox[HTML]{CCFFFF}{\makebox[\textwidth-2\fboxsep][l]{\bf - Números:}} \linebreak \linebreak
Python utiliza números inteiros e flutuantes. Pode ser utilizada a função type pra checar o valor de um objeto:\\
\begin{tabular}{l l}
\textbf{}\\
type(3) & retorna: <type 'int'> \\
type(3.14) & retorna: <type 'float'> \\
\end{tabular}

\dotfill \newline

%--------------------------------------
\colorbox[HTML]{CCFFFF}{\makebox[\textwidth-2\fboxsep][l]{\bf - Entrada de Dados:}}
%\begin{tabular}{lp{2.0cm}lp{3.0cm}|}

\begin{tabular}{lp{5.3cm}lp{3.0cm}|}
A = input() & Aguarda a entrada de caracteres armazenados em A \\
\end{tabular}
\begin{tabular}{lp{4.7cm}lp{3.0cm}|}
B = int(input()) & Aguarda a entrada de inteiros armazenados em B \\
\end{tabular}
\begin{tabular}{lp{2.6cm}lp{3.0cm}|}
A,B = map(int,input().split()) & Aguarda a entrada de inteiros  separados por espaço, armazenados em A e B respectivamente \\
\end{tabular}
\begin{tabular}{lp{3.0cm}lp{3.0cm}|}
input("Pressione ENTER") & Aguarda pressionar ENTER para prosseguir - como não declarou nenhuma variável, não irá gravar nada. \\
\end{tabular}

\vspace{0.0em} % When there are two boxes, some whitespace may need to be added if the one on the right has more content
}

%------------------------------------------------
% Lógica Básica do Python
%------------------------------------------------

\headerbox{Data Processing}{name=introduction,column=1,row=0,bottomaligned=objectives}{

%------IF--------
\colorbox[HTML]{CCFFFF}{\makebox[\textwidth-2\fboxsep][l]{\bf - if}}
\begin{itemize}\compresslist
\item if teste:\\
........\# faça algo se teste der verdadeiro\\
elif teste2\\
........\# faça algo se teste2 der verdadeiro\\
else:\newline
........\# faça algo se ambos derem falso
\end{itemize}


%------WHILE--------
\colorbox[HTML]{CCFFFF}{\makebox[\textwidth-2\fboxsep][l]{\bf - while:}}
\begin{itemize}\compresslist
\item while teste:\\
........\# enquanto verdadeiro continue fazendo algo
\end{itemize}


%------FOR--------
\colorbox[HTML]{CCFFFF}{\makebox[\textwidth-2\fboxsep][l]{\bf - for:}}
\begin{itemize}\compresslist
\item for x in sequência\\
........\# enquanto o x estiver na sequência informada\\
........\# faça algo para cada item na sequência\\
........\# a sequência pode ser uma lista,\\
........\# elementos de uma string, etc.

\item for x in range(10)\\
........\# repita algo 10 vezes (de 0 a 9)

\item for x in range(5,10)\\
........\# repita algo 5 vezes (de 5 a 9)
\end{itemize}

\colorbox[HTML]{CCFFFF}{\makebox[\textwidth-2\fboxsep][l]{\bf - Testes Lógicos}}
\linebreak \\
\begin{tabular}{l l}
10 == 10 & retorna: True \\
10 == 11 & retorna: False \\
10!= 11 & retorna: True \\
"jack" == "jack" & retorna: True \\
"jack" == "jake" & retorna: False \\
10 > 10 & retorna: False \\
10 >= 10 & retorna: True \\
"abc" >= "abc" & retorna: True \\
"abc" < "abc" & retorna: False \\
\end{tabular}
}


%------------------------------------------------
% Listas no Python
%------------------------------------------------

\headerbox{Results}{name=results,column=2,span=2,row=0}{

\colorbox[HTML]{CCFFFF}{\makebox[\textwidth-2\fboxsep][l]{\bf - Listas no Python}}
\linebreak \\
Listas são compostas por elementos de qualquer tipo (podem ser alteradas) \linebreak \\
\begin{tabular}{@{}ll@{}}
\textbf{Manipulação de Listas no Python}\\
\multicolumn{2}{l}{\cellcolor[HTML]{DDFFFF}Criação} \\
uma\_lista = [5,3,'p',9,'e'] & cria: [5,3,'p',9,'e'] \\
\multicolumn{2}{l}{\cellcolor[HTML]{DDFFFF}Acessando} \\
uma\_lista[0] & retorna: 5 \\
\multicolumn{2}{l}{\cellcolor[HTML]{DDFFFF}Fatiando} \\
uma\_lista[1:3] & retorna: [3,'p'] \\
\multicolumn{2}{l}{\cellcolor[HTML]{DDFFFF}Comprimento} \\
len(uma\_lista) & retorna: 5 \\
\multicolumn{2}{l}{\cellcolor[HTML]{DDFFFF}count( item)} \\
\multicolumn{2}{l}{Retorna quantas vezes o item foi encontrado na lista.} \\
cont(uma\_lista('p') & retorna: 1 \\
\multicolumn{2}{l}{Pode ser usado juntamente com a função while para 'andar' pelo comprimento da lista:} \\
while x < len(uma\_lista): & retorna: [3,'p']\\
\multicolumn{2}{l}{\cellcolor[HTML]{DDFFFF}Ordenar - sort()} \\
uma\_lista.sort() & retorna: [3,5,9,'e','p'] \\
\multicolumn{2}{l}{Ordenar sem alterar a lista} \\
print(sorted(uma\_lista)) & retorna: [3,5,9,'e','p'] \\
\multicolumn{2}{l}{\cellcolor[HTML]{DDFFFF}Adicionar - append(item)} \\
uma\_lista.append(37) & retorna: [5,3,'p',9,'e',37] \\
\multicolumn{2}{l}{\cellcolor[HTML]{DDFFFF}Inserir - insert(position,item)} \\
insert(uma\_lista.append(3),200) & retorna: [5,3,200,'p',9,'e'] \\
\multicolumn{2}{l}{\cellcolor[HTML]{DDFFFF}Retornar e remover - pop(position)} \\
uma\_lista.pop() & retorna: 'e' e a lista fica [5,3,'p',9] - remove o último elemento \\
uma\_lista.pop(1) & retorna: 3 e a lista fica [5,'p',9,'e'] - remove o elemento 1 \\
\multicolumn{2}{l}{\cellcolor[HTML]{DDFFFF}Remover - remove(item)} \\
uma\_lista.remove('p') & retorna: [5,3,9,'e'] \\
\multicolumn{2}{l}{\cellcolor[HTML]{DDFFFF}Inserir} \\
uma\_lista.insert(2,'z') & retorna: [5,'z',3,'p',9,'e'] - insere na posição numerada \\
\multicolumn{2}{l}{\cellcolor[HTML]{DDFFFF}Inverter - reverse()} \\
reverse(uma\_lista) & retorna: ['e',9,'p',3,5] \\
\multicolumn{2}{l}{\cellcolor[HTML]{DDFFFF}Concatenar} \\
uma\_lista+[0] & retorna: [5,3,'p',9,'e',0] \\
uma\_lista+uma\_lista & retorna: [5,3,'p',9,'e',5,3,'p',9,'e'] \\
\multicolumn{2}{l}{\cellcolor[HTML]{DDFFFF}Encontrar} \\
9 in uma\_lista & retorna: True \\
for x in uma\_lista & retorna toda a lista, um elemento por linha \\
......print(x) &  
\end{tabular}
%------------------------------------------------
}
\end{poster}

\end{document}